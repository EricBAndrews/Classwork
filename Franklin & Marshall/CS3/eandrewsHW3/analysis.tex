\documentclass[12pt]{article}
\usepackage[margin=2cm,bottom=2.5cm,top=3cm]{geometry}
\usepackage{fancyhdr}
\usepackage{setspace}
\usepackage{amssymb}
\usepackage{amsmath}
\usepackage{lineno}
\usepackage{tabularx}
\usepackage[version=4]{mhchem}
\usepackage{chemformula}
\usepackage{array}
\usepackage{multirow}
\usepackage{pgfplots}
\usepackage{endnotes}
\usepackage{listings}
\usepackage{color}
\usepackage{pgf}
\usepackage{tikz}
\usepackage{circuitikz}
\usetikzlibrary{arrows.meta}
\usetikzlibrary{calc}
\usetikzlibrary{automata,positioning}
\usetikzlibrary{shapes}

\definecolor{dkgreen}{rgb}{0,0.6,0}
\definecolor{gray}{rgb}{0.5,0.5,0.5}
\definecolor{mauve}{rgb}{0.58,0,0.82}

\lstset{language = c,
numbers = left,
tabsize = 4,
frame = single,
keywordstyle = {},
showstringspaces = false,
}
\fancyhead[LE,LO]{}
\fancyhead[CE,CO]{\textsc{Analysis of Parallel Dijkstra}}
\fancyhead[RE,RO]{}
\lfoot{}
\cfoot{\thepage}
\rfoot{}
\renewcommand{\headrulewidth}{.01cm}
\renewcommand{\footrulewidth}{.01cm}
\renewcommand\linenumberfont{\normalfont\tiny}
\renewcommand{\notesname}{Endnotes}
\newcommand{\ditto}{
    \tikz{
        \draw [line width=0.12ex] (-0.2ex,0) -- +(0,0.8ex)
            (0.2ex,0) -- +(0,0.8ex);
        \draw [line width=0.08ex] (-0.6ex,0.4ex) -- +(-1.5em,0)
            (0.6ex,0.4ex) -- +(1.5em,0);
    }
}
\newcolumntype{L}[1]{>{\raggedright\let\newline\\\arraybackslash\hspace{0pt}}m{#1}}
\newcolumntype{C}[1]{>{\centering\let\newline\\\arraybackslash\hspace{0pt}}m{#1}}
\newcolumntype{R}[1]{>{\raggedleft\let\newline\\\arraybackslash\hspace{0pt}}m{#1}}
\setlength{\parindent}{.5cm}

\begin{document}
\title{Analysis of Parallel Dijkstra}
\author{Eric Andrews}
\date{\today}
\pagestyle{fancy}
\maketitle
\onehalfspacing

\rule{\textwidth}{.1pt}
\vspace*{.75cm}

\centerline{\textbf{Initialize}}

\noindent Work:

For each vertex $v$, the algorithm must set perform a single assignment. Thus the
work of is given by:

\centerline{$v$}

\noindent Span:

No iteration requires any other; given $v$ threads, each assignment can be done
concurrently. Thus the span is given by:

\centerline{$1$}

\vspace{.75cm}

\centerline{\textbf{Get Min Vertex}}

\noindent Work:

The algorithm must check each vertex once. Thus the work of the algorithm is given
by:

\centerline{$v$}

\noindent Span:

No iteration requires any other; given $v$ threads, getMinVertex could perform
each check concurrently.

If given $v$ threads, the algorithm must merge them together to find the global min
from the local mins. This is done in a tree structure, which gives it a
logarithmic runtime. Thus the span is given by:

\centerline{$1 + \lceil\log_{2}(v)\rceil$}

\newpage

\centerline{\textbf{Update Distances}}

\noindent Work:

The algorithm must check each vertex once. Thus the work of the algorithm is given
by:

\centerline{$v$}

\noindent Span:

Each check can be performed concurrently. Thus the span of the algorithm is given
by:

\centerline{$1$}

\vspace{.75cm}

\centerline{\textbf{Parallel Dijkstra}}

\noindent Work:

initialize is performed once, and has work $v$.

getMinVertex and updateDistances are each performed $v - 1$ times, and have a
combined work of 2v. Thus the work of the overall algorithm is given by:

\begin{align*}
  & v + 2v(v - 1) \\
  &= v + 2v^{2} - 2v \\
  &= 2v^{2} - v
\end{align*}
  
\noindent Span:

initialize is performed once, and has work $1$.

getMinVertex and updateDistances are each performed $v - 1$ times in serial, and
have spans of $1 + \lceil\log_{2}(v)\rceil$ and $1$, respectively. Thus the span
of the overall algorithm is given by:

\begin{align*}
  & 1 + (v - 1)(1 + 1 + \lceil\log_{2}(v)\rceil) \\
  &= 1 + (v - 1)(2 + \lceil\log_{2}(v)\rceil) \\
\end{align*}

\newpage

\centerline{\textbf{Questions}}

\begin{enumerate}
\item
  The ideal runtime of parallel Dijkstra's is approximately
  O$(v \times \log_{2}(v))$. Therefore if $e$ is significantly greater than $v$,
  it is more efficient to use a parallel algorithm; otherwise thread overhead
  makes the adjacency list implementation more efficient. Thus in sparse graphs I
  would use an adjacency list, and in dense graphs I would use a parallel
  algorithm.

\item
  With only 1 processor, the runtime of parallel Dijkstra's is O$(v^{2} - v)$. As
  the number of processors increases, this is divided by the number of processors.
  Taking into account thread overhead, unless the graph is dense enough that $e$
  is significantly greater than $v^{2}$ it is more efficient to use an adjacency
  list implementation.

\item
  The parallel implementation iterates on all vertices for each vertex checked.
  This makes it very easy to divide into independent segments for each thread
  to operate on. The adjacency list implementation, however, iterates by the
  number of edges adjacent to the vertex currently being worked on. This makes it
  highly efficient for serial implementations, but renders it difficult to run in
  parallel, as each iteration can only be divided as many times as there are
  adjacent edges. This is both irregular---and therefore awkward to
  parallelize---and not worth the thread overhead if the graph is not
  exceptionally dense.
  
\end{enumerate}

\end{document}


%%% Local Variables:
%%% mode: latex
%%% TeX-master: t
%%% End:
